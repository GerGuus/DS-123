% Options for packages loaded elsewhere
\PassOptionsToPackage{unicode}{hyperref}
\PassOptionsToPackage{hyphens}{url}
%
\documentclass[
  12pt,
]{article}
\usepackage{amsmath,amssymb}
\usepackage{iftex}
\ifPDFTeX
  \usepackage[T1]{fontenc}
  \usepackage[utf8]{inputenc}
  \usepackage{textcomp} % provide euro and other symbols
\else % if luatex or xetex
  \usepackage{unicode-math} % this also loads fontspec
  \defaultfontfeatures{Scale=MatchLowercase}
  \defaultfontfeatures[\rmfamily]{Ligatures=TeX,Scale=1}
\fi
\usepackage{lmodern}
\ifPDFTeX\else
  % xetex/luatex font selection
\fi
% Use upquote if available, for straight quotes in verbatim environments
\IfFileExists{upquote.sty}{\usepackage{upquote}}{}
\IfFileExists{microtype.sty}{% use microtype if available
  \usepackage[]{microtype}
  \UseMicrotypeSet[protrusion]{basicmath} % disable protrusion for tt fonts
}{}
\makeatletter
\@ifundefined{KOMAClassName}{% if non-KOMA class
  \IfFileExists{parskip.sty}{%
    \usepackage{parskip}
  }{% else
    \setlength{\parindent}{0pt}
    \setlength{\parskip}{6pt plus 2pt minus 1pt}}
}{% if KOMA class
  \KOMAoptions{parskip=half}}
\makeatother
\usepackage{xcolor}
\usepackage[margin=1in]{geometry}
\usepackage{color}
\usepackage{fancyvrb}
\newcommand{\VerbBar}{|}
\newcommand{\VERB}{\Verb[commandchars=\\\{\}]}
\DefineVerbatimEnvironment{Highlighting}{Verbatim}{commandchars=\\\{\}}
% Add ',fontsize=\small' for more characters per line
\usepackage{framed}
\definecolor{shadecolor}{RGB}{248,248,248}
\newenvironment{Shaded}{\begin{snugshade}}{\end{snugshade}}
\newcommand{\AlertTok}[1]{\textcolor[rgb]{0.94,0.16,0.16}{#1}}
\newcommand{\AnnotationTok}[1]{\textcolor[rgb]{0.56,0.35,0.01}{\textbf{\textit{#1}}}}
\newcommand{\AttributeTok}[1]{\textcolor[rgb]{0.13,0.29,0.53}{#1}}
\newcommand{\BaseNTok}[1]{\textcolor[rgb]{0.00,0.00,0.81}{#1}}
\newcommand{\BuiltInTok}[1]{#1}
\newcommand{\CharTok}[1]{\textcolor[rgb]{0.31,0.60,0.02}{#1}}
\newcommand{\CommentTok}[1]{\textcolor[rgb]{0.56,0.35,0.01}{\textit{#1}}}
\newcommand{\CommentVarTok}[1]{\textcolor[rgb]{0.56,0.35,0.01}{\textbf{\textit{#1}}}}
\newcommand{\ConstantTok}[1]{\textcolor[rgb]{0.56,0.35,0.01}{#1}}
\newcommand{\ControlFlowTok}[1]{\textcolor[rgb]{0.13,0.29,0.53}{\textbf{#1}}}
\newcommand{\DataTypeTok}[1]{\textcolor[rgb]{0.13,0.29,0.53}{#1}}
\newcommand{\DecValTok}[1]{\textcolor[rgb]{0.00,0.00,0.81}{#1}}
\newcommand{\DocumentationTok}[1]{\textcolor[rgb]{0.56,0.35,0.01}{\textbf{\textit{#1}}}}
\newcommand{\ErrorTok}[1]{\textcolor[rgb]{0.64,0.00,0.00}{\textbf{#1}}}
\newcommand{\ExtensionTok}[1]{#1}
\newcommand{\FloatTok}[1]{\textcolor[rgb]{0.00,0.00,0.81}{#1}}
\newcommand{\FunctionTok}[1]{\textcolor[rgb]{0.13,0.29,0.53}{\textbf{#1}}}
\newcommand{\ImportTok}[1]{#1}
\newcommand{\InformationTok}[1]{\textcolor[rgb]{0.56,0.35,0.01}{\textbf{\textit{#1}}}}
\newcommand{\KeywordTok}[1]{\textcolor[rgb]{0.13,0.29,0.53}{\textbf{#1}}}
\newcommand{\NormalTok}[1]{#1}
\newcommand{\OperatorTok}[1]{\textcolor[rgb]{0.81,0.36,0.00}{\textbf{#1}}}
\newcommand{\OtherTok}[1]{\textcolor[rgb]{0.56,0.35,0.01}{#1}}
\newcommand{\PreprocessorTok}[1]{\textcolor[rgb]{0.56,0.35,0.01}{\textit{#1}}}
\newcommand{\RegionMarkerTok}[1]{#1}
\newcommand{\SpecialCharTok}[1]{\textcolor[rgb]{0.81,0.36,0.00}{\textbf{#1}}}
\newcommand{\SpecialStringTok}[1]{\textcolor[rgb]{0.31,0.60,0.02}{#1}}
\newcommand{\StringTok}[1]{\textcolor[rgb]{0.31,0.60,0.02}{#1}}
\newcommand{\VariableTok}[1]{\textcolor[rgb]{0.00,0.00,0.00}{#1}}
\newcommand{\VerbatimStringTok}[1]{\textcolor[rgb]{0.31,0.60,0.02}{#1}}
\newcommand{\WarningTok}[1]{\textcolor[rgb]{0.56,0.35,0.01}{\textbf{\textit{#1}}}}
\usepackage{longtable,booktabs,array}
\usepackage{calc} % for calculating minipage widths
% Correct order of tables after \paragraph or \subparagraph
\usepackage{etoolbox}
\makeatletter
\patchcmd\longtable{\par}{\if@noskipsec\mbox{}\fi\par}{}{}
\makeatother
% Allow footnotes in longtable head/foot
\IfFileExists{footnotehyper.sty}{\usepackage{footnotehyper}}{\usepackage{footnote}}
\makesavenoteenv{longtable}
\usepackage{graphicx}
\makeatletter
\def\maxwidth{\ifdim\Gin@nat@width>\linewidth\linewidth\else\Gin@nat@width\fi}
\def\maxheight{\ifdim\Gin@nat@height>\textheight\textheight\else\Gin@nat@height\fi}
\makeatother
% Scale images if necessary, so that they will not overflow the page
% margins by default, and it is still possible to overwrite the defaults
% using explicit options in \includegraphics[width, height, ...]{}
\setkeys{Gin}{width=\maxwidth,height=\maxheight,keepaspectratio}
% Set default figure placement to htbp
\makeatletter
\def\fps@figure{htbp}
\makeatother
\setlength{\emergencystretch}{3em} % prevent overfull lines
\providecommand{\tightlist}{%
  \setlength{\itemsep}{0pt}\setlength{\parskip}{0pt}}
\setcounter{secnumdepth}{-\maxdimen} % remove section numbering
\usepackage[T2A]{fontenc} \usepackage[utf8]{inputenc} \usepackage[russian]{babel}
\ifLuaTeX
  \usepackage{selnolig}  % disable illegal ligatures
\fi
\usepackage{bookmark}
\IfFileExists{xurl.sty}{\usepackage{xurl}}{} % add URL line breaks if available
\urlstyle{same}
\hypersetup{
  pdftitle={Модуль 1. Базовий. Лабораторна робота №1. Створення основи типового Data Science-проекту},
  pdfauthor={{[}Іващенко А.В.{]}},
  hidelinks,
  pdfcreator={LaTeX via pandoc}}

\title{Модуль 1. Базовий. Лабораторна робота №1. Створення основи
типового Data Science-проекту}
\author{{[}Іващенко А.В.{]}}
\date{2024-11-06}

\begin{document}
\maketitle

{
\setcounter{tocdepth}{2}
\tableofcontents
}
\subsection{2.3.1 Постановка
задачі}\label{ux43fux43eux441ux442ux430ux43dux43eux432ux43aux430-ux437ux430ux434ux430ux447ux456}

\[
y(x) = b_0 x + b_1 + b_2 x^2 \quad \text{для діапазону} \quad x \in [x_1; x_2]
\]

\begin{Shaded}
\begin{Highlighting}[]
\CommentTok{\# Задаємо параметри функції}
\NormalTok{b0 }\OtherTok{\textless{}{-}} \DecValTok{2}
\NormalTok{b1 }\OtherTok{\textless{}{-}} \DecValTok{3}
\NormalTok{b2 }\OtherTok{\textless{}{-}} \FloatTok{1.57}

\CommentTok{\# Задаємо область визначення}

\NormalTok{x }\OtherTok{\textless{}{-}} \FunctionTok{seq}\NormalTok{(}\SpecialCharTok{{-}}\DecValTok{1}\NormalTok{, }\DecValTok{1}\NormalTok{, .}\DecValTok{1}\NormalTok{)}
\NormalTok{y }\OtherTok{\textless{}{-}}\NormalTok{ b0 }\SpecialCharTok{+}\NormalTok{ b1 }\SpecialCharTok{*}\NormalTok{ x }\SpecialCharTok{+}\NormalTok{ b2 }\SpecialCharTok{*}\NormalTok{ x}\SpecialCharTok{\^{}}\DecValTok{2}

\FunctionTok{plot}\NormalTok{(x, y,}
     \AttributeTok{type =} \StringTok{"l"}\NormalTok{,}
     \AttributeTok{col =} \StringTok{"red"}\NormalTok{,}
     \AttributeTok{main =} \StringTok{"Графік функції"}\NormalTok{,}
     \AttributeTok{xlab =} \StringTok{"x"}\NormalTok{,}
     \AttributeTok{ylab =} \StringTok{"y"}
\NormalTok{     )}
\end{Highlighting}
\end{Shaded}

\begin{verbatim}
## Warning in title(...): неизвестна ширина символа 0xb3
\end{verbatim}

\begin{verbatim}
## Warning in title(...): неизвестна ширина символа 0xe0
\end{verbatim}

\begin{verbatim}
## Warning in title(...): неизвестна ширина символа 0xd0
\end{verbatim}

\begin{verbatim}
## Warning in title(...): неизвестна ширина символа 0xe4
\end{verbatim}

\begin{verbatim}
## Warning in title(...): неизвестна ширина символа 0xf6
\end{verbatim}

\begin{verbatim}
## Warning in title(...): неизвестна ширина символа 0xda
\end{verbatim}

\begin{verbatim}
## Warning in title(...): неизвестна ширина символа 0xe4
\end{verbatim}

\begin{verbatim}
## Warning in title(...): неизвестна ширина символа 0xd7
\end{verbatim}

\begin{verbatim}
## Warning in title(...): неизвестна ширина символа 0xee
\end{verbatim}

\begin{verbatim}
## Warning in title(...): неизвестна ширина символа 0xfe
\end{verbatim}

\begin{verbatim}
## Warning in title(...): неизвестна ширина символа 0x7f
\end{verbatim}

\begin{Shaded}
\begin{Highlighting}[]
\FunctionTok{points}\NormalTok{(x, y,}
       \AttributeTok{col =} \StringTok{"blue"}\NormalTok{)}
\end{Highlighting}
\end{Shaded}

\includegraphics{lab_1_files/figure-latex/unnamed-chunk-1-1.pdf}

\begin{Shaded}
\begin{Highlighting}[]
\NormalTok{df }\OtherTok{\textless{}{-}} \FunctionTok{data.frame}\NormalTok{(}\AttributeTok{x =}\NormalTok{ x, }\AttributeTok{y =}\NormalTok{ y) }\CommentTok{\# створюємо таблицю даних}

\FunctionTok{library}\NormalTok{(rio) }\CommentTok{\# підключення пакету}
\FunctionTok{export}\NormalTok{(df, }\StringTok{"data/data.csv"}\NormalTok{)}

\NormalTok{dfNew }\OtherTok{\textless{}{-}}  \FunctionTok{import}\NormalTok{(}\StringTok{"data/data.csv"}\NormalTok{)}

\CommentTok{\# Таблиця засобами knitr}
\NormalTok{knitr}\SpecialCharTok{::}\FunctionTok{kable}\NormalTok{(}\FunctionTok{head}\NormalTok{(dfNew),}
             \AttributeTok{caption =} \StringTok{"\_Табл. 1. Фрагмент таблиці даних\_"}\NormalTok{)}
\end{Highlighting}
\end{Shaded}

\begin{longtable}[]{@{}rr@{}}
\caption{\emph{Табл. 1. Фрагмент таблиці даних}}\tabularnewline
\toprule\noalign{}
x & y \\
\midrule\noalign{}
\endfirsthead
\toprule\noalign{}
x & y \\
\midrule\noalign{}
\endhead
\bottomrule\noalign{}
\endlastfoot
-1.0 & 0.5700 \\
-0.9 & 0.5717 \\
-0.8 & 0.6048 \\
-0.7 & 0.6693 \\
-0.6 & 0.7652 \\
-0.5 & 0.8925 \\
\end{longtable}

\end{document}
